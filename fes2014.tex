\documentclass{beamer}
\usepackage{tikz}
\usepackage{graphicx}
\usepackage{booktabs}
%\graphicspath{F:/antimicrobial resistance/Simon/thorax2/FES2014/R_code}
\usepackage{hyperref}

\definecolor{pheMainRed}{cmyk}{0, 1, 0.61, 0.43}
\definecolor{pheMainBlue}{cmyk}{0.9, 0, 0.49, 0}
\defbeamertemplate*{title page}{customised}[1][]{%

%-----------------------------------------------------------
% Redefine title page
%-----------------------------------------------------------
\begin{tikzpicture}[remember picture,overlay]
\fill[pheMainRed]
([yshift=60pt]current page.west) rectangle (current page.south east);
%\draw([yshift=60pt]current page.west) circle (0.5cm);
\fill[pheMainBlue]
([yshift=60pt]current page.west) rectangle ([yshift=65pt]current page.east);
\node[anchor=west]
	at ([yshift=-5pt]current page.west) (title)
%	(title)
	{\parbox[t]{\linewidth}{\raggedright\usebeamerfont{title}\usebeamercolor[fg]{title}\Large\inserttitle}};
\node[anchor=west]
	at ([yshift=25pt]current page.south west) (author)
	{\parbox[t]{\linewidth}{\raggedright\usebeamerfont{title}\usebeamercolor[fg]{title}\normalsize\insertauthor}};	
\node[anchor=south west] 
  at ([yshift=90pt]current page.west) (logo)
  {\parbox[t]{.19\paperwidth}{\raggedleft%
    \usebeamercolor[fg]{titlegraphic}\inserttitlegraphic}};
\end{tikzpicture}
}
%-----------------------------------------------------------
% Redefine footer
%-----------------------------------------------------------
\makeatother
\setbeamertemplate{footline}
{
  \leavevmode%
  \hbox{%
  \begin{beamercolorbox}[wd=.2\paperwidth,ht=2.25ex,dp=1ex,center]{footBox}%
%    \usebeamerfont{author in head/foot}\insertshortauthor
    \insertframenumber{} / \inserttotalframenumber\hspace*{1ex}
  \end{beamercolorbox}%
  \begin{beamercolorbox}[wd=.8\paperwidth,ht=2.25ex,dp=1ex,center]{footBox}%
    \usebeamerfont{title in head/foot}\insertshorttitle\hspace*{3em}

  \end{beamercolorbox}}%
  \vskip0pt%
}

%-----------------------------------------------------------
% Redefine colours
%-----------------------------------------------------------
\setbeamercolor{footBox}{fg=white, bg=pheMainRed}
\setbeamercolor{title}{fg=white}
\setbeamercolor{frametitle}{fg=pheMainBlue}
\setbeamercolor{bibliography entry author}{fg=pheMainBlue}
\setbeamercolor{bibliography entry location}{fg=white!30!pheMainBlue}
\setbeamercolor{itemize item}{fg=pheMainBlue}
\setbeamercolor{itemize subitem}{fg=pheMainBlue}
\setbeamercolor{itemize subsubitem}{fg=pheMainBlue}
\setbeamercolor{description item}{fg=pheMainBlue}

%\setbeamercolor{title}{fg=pheMainRed}
\titlegraphic{\includegraphics[width=2cm]{phe_logo}}
\setbeamercolor{titleBox}{fg=white, bg=pheMainRed}
\hypersetup{urlcolor = pheMainBlue} % Set colour of URLs

\setbeamercovered{transparent}

%-----------------------------------------------------------
% Title and authors
%-----------------------------------------------------------
\title[Trends in susceptibility of \textit{H. influenzae}, \textit{S. pneumoniae} and \textit{S. aureus} from respiratory isolates]{Trends in the antibioitic susceptibilities of \textit{Haemophilus influenzae}, \textit{Streptococcus pneumoniae} and \textit{Staphylococcus aureus} from respiratory isolates in England, Wales and Northern Ireland, 2008 - 2013}
\subtitle{}
\author{Simon Thelwall\inst{1}
\and
Rebecca Guy\inst{1}
\and Katherine L Henderson\inst{1}
\and Macey L Murray\inst{2}
\and Mike Sharland\inst{3}
\and Berit Muller-Pebody\inst{1} 
\and Alan Johnson\inst{1}}

\institute[Public Health England] % (optional, but mostly needed)
{
  \inst{1}%
  Department of Healthcare Associated Infections and Antimicrobial Resistance, Centre for Disease Surveillance and Control, Public Health England
  \\
  \inst{2}%
  Centre for Paediatric Pharmacy Research, Department of Practice and Policy, UCL School of Pharmacy, University College London, London, UK
  \\
  \inst{3}%
  Paediatric Infectious Diseases Unit, St George's Hospital NHS Trust, London, UK
}
\date{03 March 2014}

%-----------------------------------------------------------
% Start the actual document
%-----------------------------------------------------------

\begin{document}
{ % remove footline from title page. 
\setbeamertemplate{footline}{} 
\begin{frame}
\maketitle
\end{frame}
}
\addtocounter{framenumber}{-1}

\begin{frame}{Background - 1}
Resistance in organisms likely to cause secondary infections in the event of an influenza pandemic. 
\begin{thebibliography}{Blackburn et al, 2011}
\bibitem{Blackburn2011}
{\footnotesize
Blackburn, R. M., Henderson, K. L., Lillie, M., Sheridan, E., George, R. C., Deas, A. H. B., \& Johnson, A. P. (2011). 
\newblock Empirical treatment of influenza-associated pneumonia in primary care: a descriptive study of the antimicrobial susceptibility of lower respiratory tract bacteria (England, Wales and Northern Ireland, January 2007-March 2010). 
\newblock Thorax, 66(5), 389–95. doi:10.1136/thx.2010.134643
}
\end{thebibliography}
\end{frame}

\begin{frame}{Background - 2}
Recommended empiric antibiotic for community management of CAP in adults:
\begin{itemize}
\item Amoxicillin 
\item Doxycycline or clarithromycin for penicillin hypersensitive patients
\item On failure of initial empiric therapy: macrolide
\end{itemize}
\begin{thebibliography}{Lim et al, 2009}
\bibitem{Lim2009}
{\footnotesize
Lim, W. S., Baudouin, S. V, George, R. C., Hill, A. T., Jamieson, C., Le Jeune, I., … Woodhead, M. A. (2009). 
\newblock BTS guidelines for the management of community acquired pneumonia in adults: update 2009. 
\newblock Thorax, 64 Suppl 3(October), iii1–55. doi:10.1136/thx.2009.121434
}
\end{thebibliography}
\end{frame}

\begin{frame}{Methods}
\begin{description}
\item[Data source]LabBase - voluntary surveillance of clinically relevant infections. 
\begin{itemize}
\item Respiratory samples: sputum, trachea, broncheoles or alveolar lavage.
\item January 2008 - June 2013.
\item hospital outpatient or GP patients.
\end{itemize}
\pause \item[Resistance] Isolate reported as intermediate or resistant. \hfill \\
Where multiple sensitivities reported, resistant result took precendence.
\pause \item[Analysis] R version 3.0.2
\begin{itemize}
\pause \item Linear regression for trend in number of isolates reported.
\pause \item Poisson regression for annual change in number of resistant isolates reported.
\begin{itemize}
\item Adjusted for age, sex, calendar quarter and region.
\end{itemize}
\end{itemize}
\end{description}
\end{frame}
\section{Results}
\begin{frame}{Results - summary figures}
\begin{itemize}
\item 180 499 isolates
\begin{itemize}
\item 117 579 (65 \%) \textit{H. influenzae}
\item 31 914 (18 \%) \textit{S. aureus}
\item 31 006 (17 \%) \textit{S. pneumoniae}
\end{itemize}
\pause \item 142 labs
\pause \item 96 \% from sputum
\pause \item 70 \% from GP patients
\end{itemize}
\end{frame}

\begin{frame}{Results - age structure}
\begin{figure}
\includegraphics[width = \textwidth]{age_fig}
%\caption{Age structure for patients providing respiratory isolates, by organism. England, Wales and Northern Ireland. 2008 - June 2013.}
\end{figure}
\end{frame}

\begin{frame}{Results - Trend in number of isolates}
\begin{figure}
\includegraphics[width = \textwidth]{trend_isolates}
\end{figure}
\end{frame}

\begin{frame}{Results - Trend in proportion of isolates tested}
\begin{figure}
\includegraphics[width = \textwidth]{trend_testing}
\end{figure}
\end{frame}

\begin{frame}{Results - Trend in resistance} 
\begin{figure}
\includegraphics[width = \textwidth]{trend_resistance}
\end{figure}
\end{frame}

\begin{frame}{Results - Incident rate ratios for annual change in susceptibility}
% The following is an enormous mess but gets the job done. 
\begin{table}
   \centering
   \tiny%\addtolength{\tabcolsep}{-5pt}
   \begin{tabular}{rrlll}
  \hline
 &  & \multicolumn{3}{c}{Organism}\\
  \cline{3-5}
Antibiotic & \shortstack{Age \\group} & \textit{H. influenzae} & \textit{S. aureus} & \textit{S. pneumoniae} \\ 
  \midrule
\onslide<1>{ \shortstack{Ampicillin/\\amoxicillin}} & \onslide<1>{ - }& \onslide<1>{ 1.03 (1.02-1.03)} & \onslide<1>{ 1.01 (1.00-1.03)} & \onslide<1>{ 1.12 (1.05-1.20)} \\ 
\onslide<2>{ Clarithromycin} & \onslide<2>{$<45$ }& \onslide<2>{1.01 (1.00-1.01)} & \onslide<2>{1.00 (0.95-1.06)} & \onslide<2>{1.06 (0.96-1.17)}\\ 
\onslide<2>{  }& \onslide<2>{ 45-64} & \onslide<2>{ 1.01 (1.00-1.01)} & \onslide<2>{ 1.01 (0.95-1.07)} & \onslide<2>{ 1.10 (1.03-1.16)} \\ 
\onslide<2>{  }& \onslide<2>{ 65-74} & \onslide<2>{ 1.00 (1.00-1.01)} & \onslide<2>{ 0.95 (0.89-1.01)} & \onslide<2>{ 1.11 (1.04-1.18)} \\ 
\onslide<2>{  }& \onslide<2>{ $\geq 75$} & \onslide<2>{ 1.01 (1.00-1.01)} & \onslide<2>{ 1.05 (0.99-1.12)} & \onslide<2>{ 1.09 (1.01-1.17)} \\ 
\onslide<2>{  }& \onslide<2>{ Unknown} & \onslide<2>{ 1.00 (0.91-1.09)} & \onslide<2>{ 0.97 (0.09-10.75)} & \onslide<2>{ 0.99 (0.70-1.40)} \\ 
\onslide<3>{ Doxycycline} & \onslide<3>{$<45$} & \onslide<3>{0.80 (0.65-0.98)} & \onslide<3>{1.36 (1.13-1.63)} & \onslide<3>{1.03 (0.81-1.31)} \\ 
\onslide<3>{  }& \onslide<3>{45-64} & \onslide<3>{0.85 (0.71-1.01) }& \onslide<3>{1.01 (0.85-1.20)} & \onslide<3>{1.14 (1.01-1.28)} \\ 
\onslide<3>{  }& \onslide<3>{65-74} & \onslide<3>{0.86 (0.69-1.07) }& \onslide<3>{1.18 (0.90-1.55)} & \onslide<3>{1.34 (1.18-1.52)} \\ 
\onslide<3>{  }& \onslide<3>{$\geq 75$} & \onslide<3>{0.90 (0.71-1.15)} & \onslide<3>{1.14 (0.90-1.42)} & \onslide<3>{1.10 (0.94-1.29)} \\ 
\onslide<3>{  }& \onslide<3>{Unknown} & \onslide<3>{0.93 (0.57-1.54)} & \onslide<3>{0.95 (0.08-10.74)} & \onslide<3>{- }\\ 
\bottomrule  
\end{tabular}
%   \caption{Incident rate ratios for annual change in susceptibility to recommended antimicrobials, England, Wales and Northern Ireland 2008-2013.
%   Adjusted for calendar quarter, laboratory region, specimen source, male sex and patient age.}
%   \label{tab:table1}
   \end{table} 
\end{frame}

\begin{frame}{Trend in resistance to both ampicillin/amoxicillin and a macrolide}
\begin{figure}
\includegraphics[width = \textwidth]{trend_multi}
\end{figure}
\end{frame}

\begin{frame}{Incident rate ratios for multi-resistance.}
\begin{figure}
incident rate ratio table in here. Interactions?
\end{figure}
\end{frame}

\section{Conclusions}
\begin{frame}{Conclusions}
\begin{itemize}
\item Increasing numbers of isolates being reported. 
\pause \item Increasing resistance to recommended antibiotics being observed in some organisms.
\pause \item Resistance to first and second line treatments increasing. 
\end{itemize}
\end{frame}

\begin{frame}{End matter}
\begin{description}
\item[Acknowledgements]\insertinstitute
\item[Code] \url{https://github.com/simonthelwall/}
\item[Email] \href{mailto:simon.thelwall@phe.gov.uk}{\nolinkurl{simon.thelwall@phe.gov.uk} }
\end{description}
\end{frame}

\end{document}
